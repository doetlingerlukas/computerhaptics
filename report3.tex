\documentclass[11pt,a4paper]{article}

\usepackage{lmodern}
\usepackage{amsmath}
\usepackage{amssymb}
\usepackage[english]{babel}
\usepackage[procnames]{listings}
\usepackage{listings}
\usepackage{ulem}
\usepackage{amsthm}
\usepackage{tikz}
\usepackage{float}
\usepackage{wasysym}
\usepackage[binary-units=true]{siunitx}
\usepackage{pgfplots}
\DeclareSIPrefix\nona{N}{10^27}

\usepackage{multicol}
\usepackage{enumitem}
\setenumerate{nolistsep, itemsep=.5em}
\setitemize{nolistsep, itemsep=.5em}

\setlength{\parindent}{0em}
\setlength{\parskip}{.5em}

\author{Lukas Dötlinger}
\date{}

\definecolor{red}{HTML}{f92672}
\definecolor{green}{HTML}{009900}

\pgfplotsset{compat=1.17}

\pgfplotsset{every tick label/.append style={font=\tiny}}

\begin{document}
  \title{Computer Haptics: Assignment 2}
  \maketitle

  \section*{Task 1 - P}

  To realise a \textit{P-Type} controller, the force was calculated by:
  \begin{equation*}
    F = K_p * error + 0.7 * sgn(error)
  \end{equation*}
  The error is measured in degrees and calculated by subtracting the reference-input of 20 degrees from the current angle. To counter the friction of the \textit{Hapkit}, an additional \textit{"base force"} of $0.7$ is added. The \texttt{sgn()} function is used to determine the correct direction for that force. The following graph shows the force development over time for three different values. The time $t$ is measured in milliseconds.

  \begin{figure}[H]
    \centering
    \begin{tikzpicture}
      \begin{axis}[
        height=7cm,
        width=\textwidth,
        xlabel={$t$},
        ylabel={F},
        xmin=0, xmax=2000,
        ymin=-15, ymax=15,
        xtick={0,500,1000,1500},
        ytick={-10,-5,-2,0,2,5,10},
        xtick pos=left,
        ytick pos=left,
      ]

        \addplot [ smooth, very thick, color=cyan ] table {res/measurements-3-1-1.txt};

        \addplot [ smooth, thick, color=orange ] table {res/measurements-3-1-2.txt};

        \addplot [ smooth, color=red ] table {res/measurements-3-1-3.txt};

        \legend{$K_p = 0.15$,$K_p = 0.35$,$K_p = 0.70$}

      \end{axis}
    \end{tikzpicture}
  \end{figure}

  The cyan plot shows a damped response, while the orange one resides in a state of continuous vibration. The red plot visualises an unstable response.

  \section*{Task 2 - PD}

  To realise a \textit{PD-Type} controller, the force was calculated by:
  \begin{equation*}
    F = (K_p * error + 0.4 * sgn(error)) - K_d * v
  \end{equation*}
  To dampen the response we multiply the velocity $v$ with $K_d$ and subtract it. The velocity of the handle was used, as it's equal to the error change over time, $\frac{d}{dt}$. In comparison to the first task, we are also using a lower \textit{"base force"}. The following graph shows the force development over time for three different value-pairs.

  \begin{figure}[H]
    \centering
    \begin{tikzpicture}
      \begin{axis}[
        height=7cm,
        width=\textwidth,
        xlabel={$t$},
        ylabel={F},
        xmin=0, xmax=2000,
        ymin=-1.25, ymax=1,
        xtick={0,500,1000,1500},
        ytick={-1,-0.5,0,0.5,1},
        xtick pos=left,
        ytick pos=left,
      ]

        \addplot [ smooth, very thick, color=cyan ] table {res/measurements-3-2-1.txt};

        \addplot [ smooth, thick, color=blue ] table {res/measurements-3-2-2.txt};

        \addplot [ smooth, thick, color=magenta ] table {res/measurements-3-2-3.txt};

        \legend{$K_d = 0.10$ $K_p = 0.0250$,$K_d = 0.15$ $K_p = 0.0175$,$K_d = 0.15$ $K_p = 0.0400$}

      \end{axis}
    \end{tikzpicture}
  \end{figure}

  When comparing the settling-time to the previous graph, we can observe a clear reduction. The sudden increase in force at the magenta-coloured plot results from the applied \textit{"base force"}, as we slightly move past the desired angle and therefore get a changed sign of the force. The \textit{"base force"} is also the reason for the paddle not idling at zero. This is due to the imprecision of the device, since the desired angle of 20 degrees is never hit.

  \section*{Task 3 - PID}

  To realise a \textit{PID-Type} controller, the force was calculated by:
  \begin{equation*}
    F = (K_p * error + 0.4 * sgn(error)) - K_d * v + K_i * error_{sum}
  \end{equation*}
  The sum of errors $error_{sum}$ is calculated by adding the current error, multiplied with the time-difference to the last measured error in milliseconds, to the existing sum for every iteration. While this resulted in the desired effect, as shown below, the paddle was found to be unstable after a few seconds. This is due to the fact that the paddle is not exactly at the desired angle and therefore increases the error-sum over time. This results in unstable movement.

  \begin{figure}[H]
    \centering
    \begin{tikzpicture}
      \begin{axis}[
        height=7cm,
        width=\textwidth,
        xlabel={$t$},
        ylabel={F},
        xmin=0, xmax=2000,
        ymin=-2, ymax=2,
        xtick={0,500,1000,1500},
        ytick={-2,-1,0,1,2},
        xtick pos=left,
        ytick pos=left,
      ]

        \addplot [ smooth, very thick, color=cyan ] table {res/measurements-3-3-1.txt};

        \addplot [ smooth, thick, color=blue ] table {res/measurements-3-3-2.txt};

        \legend{$K_i = 0.00001$ $K_d = 0.05$ $K_p = 0.025$,$K_i = 0.00003$ $K_d = 0.05$ $K_p = 0.035$}

      \end{axis}
    \end{tikzpicture}
  \end{figure}

\end{document}