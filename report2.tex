\documentclass[11pt,a4paper]{article}

\usepackage{lmodern}
\usepackage{amsmath}
\usepackage{amssymb}
\usepackage[english]{babel}
\usepackage[procnames]{listings}
\usepackage{listings}
\usepackage{ulem}
\usepackage{amsthm}
\usepackage{tikz}
\usepackage{float}
\usepackage{wasysym}
\usepackage[binary-units=true]{siunitx}
\usepackage{pgfplots}
\DeclareSIPrefix\nona{N}{10^27}

\usepackage{multicol}
\usepackage{enumitem}
\setenumerate{nolistsep, itemsep=.5em}
\setitemize{nolistsep, itemsep=.5em}

\setlength{\parindent}{0em}
\setlength{\parskip}{.5em}

\author{Lukas Dötlinger}
\date{}

\definecolor{red}{HTML}{f92672}
\definecolor{green}{HTML}{009900}

\pgfplotsset{compat=1.17}

\pgfplotsset{every tick label/.append style={font=\tiny}}

\begin{document}
  \title{Computer Haptics: Assignment 2}
  \maketitle

  \section*{Task 1 - Spring}

  For this experiment, the absolute position $x_{user}$ was multiplied with the stiffness factor $f$ to calculate the Force. The following graph shows the output when moving the handle in one direction. The displayed results were generated with $f = 0.002$.

  \begin{figure}[H]
    \centering
    \begin{tikzpicture}
      \begin{axis}[
        xlabel={$x_{user}$},
        ylabel={F},
        xmin=0, xmax=2300,
        ymin=0, ymax=4.50,
        xtick={500,1000,1500,2000},
        ytick={1,2,3,4},
        xtick pos=left,
        ytick pos=left,
      ]

        \addplot [ only marks, color=cyan ] table {res/measurements-2-1.txt};

      \end{axis}
    \end{tikzpicture}
  \end{figure}

  Increasing the stiffness resulted in the expected instability of the device, which started to show at $f = 0.008$. This can be noticed by the paddle bouncing from left to right without any interaction, due to imprecision of the \textit{Hapkit}. The high force moves the paddle beyond the zero position into the \textit{"opposite spring"}.

  \section*{Task 3 - Virtual Wall}

  The virtual wall was rendered at position $x_{wall} = 5mm$. The force was calculated by multiplying the distance past the wall with the constant $k$:
  \begin{equation*}
    F = k * (x_{wall} - x_{user})
  \end{equation*}
  The following graph shows the applied force, when moving past the border of the wall, for $k = 0.75$.

  \begin{figure}[H]
    \centering
    \begin{tikzpicture}
      \begin{axis}[
        xlabel={$x_{user}$},
        ylabel={F},
        xmin=0, xmax=1000,
        ymin=-100, ymax=500,
        xtick={250,500,750},
        ytick={0,100,200,300,400},
        xtick pos=left,
        ytick pos=left,
      ]

        \addplot [ only marks, color=cyan ] table {res/measurements-2-3.txt};

        \addplot [ no marks, smooth, color=red ] coordinates { (340,-100) (340,500) } node[below] {$x_{wall}$};

      \end{axis}
    \end{tikzpicture}
  \end{figure}

  For $k > 1.25$, the device was found to be unstable as the resulting Force $F \approx 12.5$ created to much pressure on the rather fragile \textit{Hapkit}.

\end{document}