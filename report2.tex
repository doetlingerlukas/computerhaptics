\documentclass[11pt,a4paper]{article}

\usepackage{lmodern}
\usepackage{amsmath}
\usepackage{amssymb}
\usepackage[english]{babel}
\usepackage[procnames]{listings}
\usepackage{listings}
\usepackage{ulem}
\usepackage{amsthm}
\usepackage{tikz}
\usepackage{float}
\usepackage{wasysym}
\usepackage[binary-units=true]{siunitx}
\usepackage{pgfplots}
\DeclareSIPrefix\nona{N}{10^27}

\usepackage{multicol}
\usepackage{enumitem}
\setenumerate{nolistsep, itemsep=.5em}
\setitemize{nolistsep, itemsep=.5em}

\setlength{\parindent}{0em}
\setlength{\parskip}{.5em}

\author{Lukas Dötlinger}
\date{}

\definecolor{red}{HTML}{f92672}
\definecolor{green}{HTML}{009900}

\pgfplotsset{compat=1.17}

\pgfplotsset{every tick label/.append style={font=\tiny}}

\begin{document}
  \title{Computer Haptics: Assignment 2}
  \maketitle

  \section*{Task 1 - Spring}

  For this experiment, the absolute position $x_{user}$ was multiplied with the stiffness factor $f$ to calculate the Force. The following graph shows the output when moving the handle in one direction. The displayed results were generated with $f = 0.002$.

  \begin{figure}[H]
    \centering
    \begin{tikzpicture}
      \begin{axis}[
        xlabel={$x_{user}$},
        ylabel={F},
        xmin=0, xmax=2300,
        ymin=0, ymax=4.50,
        xtick={500,1000,1500,2000},
        ytick={1,2,3,4},
        xtick pos=left,
        ytick pos=left,
      ]

        \addplot [ only marks, color=cyan ] table {res/measurements-2-1.txt};

      \end{axis}
    \end{tikzpicture}
  \end{figure}

  Increasing the stiffness resulted in the expected instability of the device, which started to show at $f = 0.008$. This can be noticed by the paddle bouncing from left to right without any interaction, due to imprecision of the \textit{Hapkit}. The high force moves the paddle beyond the zero position into the \textit{"opposite spring"}.

  \section*{Task 2 - Friction}

  Friction was rendered in two ways, coulomb and viscous. For \textit{Coulomb Friction}, force is applied against the movement direction of the handle in a constant way. Meaning no force is applied if velocity $v(t) = 0$. For a velocity $v(t) \neq 0$, the force is calculated by
  \begin{equation*}
    F_{coulomb} = -b * sgn(v(t))
  \end{equation*},
  where $b$ is the constant defining the friction. The following graph shows an experiment with coulomb friction, with $b = 1$:

  \begin{figure}[H]
    \centering
    \begin{tikzpicture}
      \begin{axis}[
        xlabel={$v(t)$},
        ylabel={F},
        xmin=-5, xmax=5,
        ymin=-2.5, ymax=2.5,
        xtick={-4,-2,0,2,4},
        ytick={-2,-1,0,1,2},
        xtick pos=left,
        ytick pos=left,
      ]

        \addplot [ only marks, color=cyan ] table {res/measurements-2-2-1.txt};

      \end{axis}
    \end{tikzpicture}
  \end{figure}

  \textit{Viscous Friction} on the other hand, acts similar if the velocity is zero, but renders and increasing friction for increasing velocity. This force is calculated as
  \begin{equation*}
    F_{viscous} = -b * v(t)
  \end{equation*}
  and has been used for the experiment visualized in the following graph, with $b = 0.75$:

  \begin{figure}[H]
    \centering
    \begin{tikzpicture}
      \begin{axis}[
        xlabel={$v(t)$},
        ylabel={F},
        xmin=-5, xmax=6,
        ymin=-5, ymax=5,
        xtick={-4,-2,0,2,4,5},
        ytick={-4,-2,0,2,4},
        xtick pos=left,
        ytick pos=left,
      ]

        \addplot [ very thick, color=cyan ] table {res/measurements-2-2-2.txt};

      \end{axis}
    \end{tikzpicture}
  \end{figure}

  Due to some imprecision, the velocity is only computed into force if $v(t) > 0.2$, as the device would become unstable otherwise.

  \section*{Task 3 - Virtual Wall}

  The virtual wall was rendered at position $pos_{wall} = 5mm$, which was calculated to an absolute position $x_{wall}$ at angle $\alpha \approx 4.76$. The force was calculated by multiplying the distance past the wall with the constant $k$:
  \begin{equation*}
    F = k * (x_{wall} - x_{user})
  \end{equation*}
  The following graph shows the applied force, when moving past the border of the wall, for $k = 0.75$.

  \begin{figure}[H]
    \centering
    \begin{tikzpicture}
      \begin{axis}[
        xlabel={$x_{user}$},
        ylabel={F},
        xmin=0, xmax=1000,
        ymin=-100, ymax=500,
        xtick={250,500,750},
        ytick={0,100,200,300,400},
        xtick pos=left,
        ytick pos=left,
      ]

        \addplot [ only marks, color=cyan ] table {res/measurements-2-3.txt};

        \addplot [ no marks, smooth, color=red ] coordinates { (340,-100) (340,500) } node[below] {$x_{wall}$};

      \end{axis}
    \end{tikzpicture}
  \end{figure}

  For $k > 1.25$, the device was found to be unstable as the resulting Force $F \approx 12.5$ created to much pressure on the rather fragile \textit{Hapkit}.

  \section*{Task 5 - Textures}

  To simulate textures, the force $F$ for velocity $v$ was calculated using a damper $b$ with the following formula, if the position is inside a damping area:
  \begin{equation*}
    F_{v} = -b_{v} * v_{t}
  \end{equation*}
  The damper itself was set to a value of 1 to minimize vibrations, that may occur due to the discontinuity in force.

  \begin{figure}[H]
    \centering
    \begin{tikzpicture}
      \begin{axis}[
        xlabel={$x_{user}$},
        ylabel={F},
        xmin=-3000, xmax=3000,
        ymin=-2, ymax=2,
        xtick={-2500,-1500,-500,0,500,1500,2500},
        ytick={-2,-1,0,1,2},
        xtick pos=left,
        ytick pos=left,
      ]

        \addplot [ only marks, color=cyan ] table {res/measurements-2-5.txt};

      \end{axis}
    \end{tikzpicture}
  \end{figure}

  \section*{Resources}

  Apart from the provided references on the assignment sheet, the example-code and instructions given at the website of the \textit{Hapkit} project,\\ \texttt{hapkit.stanford.edu}, were used to solve this assignment.

\end{document}